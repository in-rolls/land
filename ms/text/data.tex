\section{Bihar land records}
The foundational data we use for this study is the Bihar land records. 
In total, we have around 41.87 million land records and 12.12 million individuals/accounts.
Each record is for a plot of land enumerating details such as the land owner's registered account number, name of the owner (ryot or raiyat), name of the owner's father, residence (is this village?), caste, amongst others. 
See \cref{fig:example_land_record} for an example copy of a land record and \cref{tab:fields_land_records} for the fields included in each record. 

\begin{figure}
	\centering
	\begin{subfigure}{\textwidth}
		\centering
		\includegraphics[width=\textwidth]{example_land_record_hindi.png}
		\caption{Original}
	\end{subfigure}
	\hfill
	\begin{subfigure}{\textwidth}
		\centering
		\includegraphics[width=\textwidth]{example_land_record_english.png}
		\caption{Google translated}
	\end{subfigure}
	\caption{Example of Bihar land record. Screenshot of \href{http://land.bihar.gov.in/Ror/ViewRoRReportNew.aspx?SID=xnu01z5jvad5ozksexfvtgyx&LHID=3601010338000344&KNo=108}{originating source} on \url{http://land.bihar.gov.in}.}
	\label{fig:example_land_record}
\end{figure}

To compare land area for each land plot, we use the fields 6--8 (\cref{tab:fields_land_records}) for acre, decimal, and hectare as land area in acres $= Acre + \frac{Decimal}{100} + Hectare \times 2.4711$.
Three account holders across fourteen land records have negative land plot areas and we drop these for analyses relating to land plot areas.

The original names are in Hindi. We use the Indicate package \citep{indicate} to transliterate the Hindi scripts to English.

The land records also come with caste of the owner, we work only on the top 200...


\begin{table}[t]\footnotesize \setlength\tabcolsep{6 pt} \centering
	\caption{Bihar land record fields}
	\label{tab:fields_land_records}
	\begin{tabular}{@{\hspace{0\tabcolsep}}ll@{\hspace{0\tabcolsep}}}
		\toprule
		\textbf{Field} & \textbf{Description}                              \\
		\midrule
		name of ryot   & Name of land plot owner (ryot or raiyat) \\
		name of father & Name of owner's father                   \\
		residence      &                                          \\
		caste          &                                          \\
		revenue unit   &                                          \\
		district       &                                          \\
		divison        &                                          \\
		mouza          &                                          \\
		account no     & Owner's unique account number            \\
		link           & Original link of land record             \\
		1              &                                          \\
		2              &                                          \\
		3              &                                          \\
		4              &                                          \\
		5              &                                          \\
		6              & Land plot area in acre                   \\
		7              & Land plot area in decimal                \\
		8              & Land plot area in hectare                \\
		9              &                                          \\
		10             &                                          \\
		11             &                                          \\
		12             &                                          \\
		13             &                                          \\
		14             &                                          \\
		\bottomrule
	\end{tabular}
\end{table}



\begin{SCfigure}
	\centering
	\includegraphics[width=.75\textwidth]{30_most_common_lastnames_accountholders}
	\caption{30 most common last names. Figure reports the 30 most common (transliterated) last names in the Bihar land records. Religion comes from the Hindi names and the Pranaam package \citep{pranaam}.
	}
	\label{fig:example_land_record}
\end{SCfigure}
